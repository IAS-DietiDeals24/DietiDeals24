\chapter{Glossario}
    \section{Obiettivo}
        In questa sezione sono definiti i vocaboli meno comuni utilizzati all'interno della documentazione.

    \section{Glossario}
        \begin{tabular}{|C{3.5cm}|L{11.2cm}|}
            \hline
            \multicolumn{1}{|c|}{\textbf{Vocabolo}} & \multicolumn{1}{c|}{\textbf{Definizione}}\\
            \hline
                Stakeholders &
                Insieme dei soggetti che hanno un interesse nei confronti di un'organizzazione e che con il loro comportamento possono influenzarne l'attività. Nel caso dell'ingegneria del software, tutti gli individui interessati alla messa in opera del sistema.\\
            \hline
                Requirement Elicitation (Raccolta dei Requisiti) &
                Intervista agli stakeholders per carpire informazioni sul dominio del problema, nonché le funzionalità e le qualità che il software debba possedere per risultare conforme ai bisogni dell'utente.\\
            \hline
                Requirement Analysis (Analisi dei Requisiti) &
                Le informazioni ottenute attraverso la raccolta sono organizzate in un documento attraverso l'uso di diagrammi ad alto livello, quali diagramma UML dei casi d'uso, tabelle di Cockburn e mock-up.\\
            \hline
                Loggato & Utente che ha effettuato il login , ovvero l'accesso al proprio account. \\
            \hline
        \end{tabular}