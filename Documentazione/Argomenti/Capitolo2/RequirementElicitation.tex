%PDF DI RIFERIMENTO: 03_Requirement Engineering.pdf

\chapter{Requirement Elicitation}
     \section{Introduzione al sistema} %Requisiti sistema. 1 pagina
        Sulla base dalle richieste definite dagli stakeholders sono stati individuati i seguenti requisiti di sistema:
        
        \begin{itemize}
            \item[1]
                \begin{itemize}
                    \item L'utente può registrarsi con email correttamente formattata e password.
                    \item Il sistema controlla che l'email abbia il formato corretto (ovvero sia del tipo x+@y+.z+) e che la password sia sicura (ovvero sia lunga almeno 8 caratteri e contenga, nel complesso, almeno una lettera maiuscola [A-Z], un numero [0-9] e un carattere speciale [!@\#\$\%\^{}\&*])
                    \item L'utente può registrarsi con altri account già esistenti come Google, Facebook, GitHub e X.
                    \item L'utente può selezionare se il suo nuovo account è di tipo compratore o venditore.
                    \item L'utente può creare uno e un solo account di tipo compratore con la stessa email.
                    \item L'utente può creare uno e un solo account di tipo venditore con la stessa email.
                    \item L'utente può accedere a un account con l'email e password utilizzati nella registrazione.
                    \item L'utente può accedere con altri account già esistenti come Google, Facebook, GitHub e X.
                    \item L'utente che ha effettuato l'accesso può visualizzare il suo profilo.
                    \item L'utente che ha effettuato l'accesso può modificare i dati sul suo profilo, come biografia, link al proprio sito, link social e area geografica.
                \end{itemize}
            \item[2] 
                \begin{itemize}
                    \item Il venditore può creare aste di vendita di un prodotto/servizio. L'asta risulterà attiva (cioè sarà possibile presentare offerte) fin da subito.
                    \item Il venditore può selezionare il tipo di asta da creare. L'asta può essere di tipo "a tempo fisso" o "silenziosa".
                    \item Il venditore di un'asta di qualsiasi tipo può inserire una data e ora di scadenza per l'asta da lui creata.
                    \item Il venditore può inserire/modificare la descrizione dell'asta da lui creata.
                    \item Il venditore può opzionalmente inserire/modificare la fotografia dell'oggetto in vendita nell'asta da lui creata.
                    \item Il venditore può inserire/modificare la categoria dell'asta da lui creata.
                    \item Il compratore può presentare offerte per le aste correntemente attive e gestite da venditori.
                \end{itemize}
            \item[3]
                \begin{itemize}
                    \item L'utente può ricercare le aste attraverso parole chiave.
                    \item L'utente può ricercare le aste sulla base della categoria.
                    \item L'utente può visualizzare i dettagli di un'asta. \textbf{??????}
                    \item L'utente può visualizzare il profilo del proprietario dell'asta. \textbf{??????}
                \end{itemize}
            \item[4]
                \begin{itemize}
                    \item Il venditore di un asta di tipo "a tempo fisso" può stabilire una soglia minima segreta per l'asta da lui creata (mai visibile ad altri utenti al di fuori del creatore dell'asta e deve essere un valore maggiore o uguale a 0).
                    \item L'utente può visualizzare l'attuale offerta più alta dell'asta di tipo "a tempo fisso".
                    \item Il sistema assegna la vincita al compratore di un asta di tipo "a tempo fisso" che ha offerto la somma più alta entro la data e ora di scadenza dell'asta.
                    \item Il sistema dichiara l'asta fallita se nessun compratore di un asta di tipo "a tempo fisso" ha offerto una somma più alta della soglia minima entro la data e ora di scadenza dell'asta.
                    \item Il sistema dichiara l'asta fallita se non è stata avanzata alcuna offerta da parte di un compratore di un asta di tipo "a tempo fisso" entro la data e ora di scadenza dell'asta.
                    \item Il sistema invia una notifica a tutti i partecipanti dell'asta di tipo "a tempo fisso" (ovvero il venditore e tutti i compratori che hanno offerto almeno una volta una somma di denaro) entro 10 secondi dalla data e ora di scadenza dell'asta.
                \end{itemize}
            \item[7] 
                \begin{itemize}
                    \item Il compratore di un'asta di tipo "silenziosa" può presentare un'offerta segreta al venditore di tale asta (cioè visualizzabile solo a tale venditore e a colui che ha proposto l'offerta), entro la data e ora di scadenza dell'asta.
                    \item Il venditore di un'asta di tipo "silenziosa" può accettare una e una sola offerta, inviata tra le offerte non ancora rifiutate dell'asta di tipo "silenziosa" da lui creata.
                    \item Il venditore di un'asta di tipo "silenziosa" può rifiutare le singole offerte ricevute all'asta di tipo "silenziosa" da lui creata. \textbf{??????}
                    \item Il sistema assegna la vincita dell'asta al compratore di un'asta di tipo "silenziosa" la cui offerta è stata accettata dal venditore dell'asta di tipo "silenziosa".
                    \item Il sistema dichiara l'asta fallita se il venditore di un'asta di tipo "silenziosa" non ha accettato alcuna offerta entro la data e ora di scadenza dell'asta.
                    \item Il sistema dichiara l'asta fallita se non è stata avanzata alcuna offerta da parte di un compratore di un'asta di tipo "silenziosa" entro la data e ora di scadenza dell'asta.
                \end{itemize}
            \item[8]
                \begin{itemize}            
                    \item Il compratore può creare un'asta di tipo "inversa".
                    \item Il compratore di un'asta di tipo "inversa" può inserire una data e ora di scadenza per l'asta di tipo "inversa" da lui creata.
                    \item Il compratore di un'asta di tipo "inversa" può specificare il prodotto/servizio che richiede nell'asta di tipo "inversa" da lui creata.
                    \item Il compratore di un'asta di tipo "inversa" può opzionalmente inserire/modificare un'immagine del prodotto/servizio specificato nell'asta di tipo "inversa" da lui creata.
                    \item Il compratore di un'asta di tipo "inversa" può inserire un prezzo minimo che è disposto a pagare per il prodotto/servizio dell'asta di tipo "inversa" da lui creata.
                    \item Il compratore di un'asta di tipo "inversa" può inserire/modificare la categoria dell'asta di tipo "inversa" da lui creata.
                    \item Il venditore di un'asta di tipo "inversa" in grado di fornire quel particolare prodotto/servizio può presentare offerte per le aste di tipo "inversa" correntemente attive.
                    \item Il venditore di un'asta di tipo "inversa" può offrire una somma in euro minore rispetto alla somma più bassa attualmente raggiunta, entro la data e ora di scadenza dell'asta di tipo "inversa".
                    \item Il sistema assegna la vincita dell'asta di tipo "inversa" al venditore che ha offerto la somma più bassa al sopraggiungere della data e ora di scadenza dell'asta a cui ha partecipato.
                    \item Il sistema dichiara l'asta di tipo "inversa" fallita se non è stata avanzata alcuna offerta da parte di un venditore entro la data e ora di scadenza dell'asta.
                \end{itemize}
        \end{itemize}