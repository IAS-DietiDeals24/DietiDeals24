%PDF DI RIFERIMENTO: 03_Requirement Engineering.pdf

\chapter{Requirement Elicitation}
     \section{Introduzione al sistema} %Requisiti sistema. 1 pagina
        Sulla base dalle richieste definite dagli stakeholders sono stati individuati i seguenti requisiti di sistema:
        
        \begin{itemize}
            \item[1]
            \begin{itemize} % Requisiti %
                \item L'utente può registrarsi con email e password.
                \item L'utente può registrarsi con altri account già esistenti come Google, Facebook, GitHub e altri social.
                \item L'utente può selezionare se il suo nuovo account è di tipo acquirente o venditore.
                \item L'utente può creare uno e un solo account di tipo acquirente.
                \item L'utente può creare uno e un solo account di tipo venditore.
                \item L'utente può accedere con email e password.
                \item L'utente può accedere con altri account già esistenti come Google, Facebook, GitHub e altri social.
                \item L'utente può visualizzare il suo profilo.
                \item L'utente può modificare i dati sul suo profilo, come biografia, link al proprio sito, link social, area geografica ed altri aspetti.
            \end{itemize} %%%
            \item[2] 
            \begin{itemize} % Requisiti %
                \item Il venditore può creare un'asta di vendita di un prodotto/servizio.
                \item Il venditore può presentare offerte per le aste correntemente attive e gestite da compratori.
                \item Il venditore può inserire/modificare la descrizione dell'asta da lui gestita.
                \item Il venditore può opzionalmente inserire/modificare la fotografia dell'oggetto in vendita nell'asta da lui gestita.
                \item Il venditore può inserire/modificare la categoria dell'asta da lui gestita.
                \item Il compratore può presentare offerte per le aste correntemente attive e gestite da venditori.
            \end{itemize} %%%
            \item[3]
            \begin{itemize} % Requisiti %
                \item L'utente può ricercare un'asta attraverso parole chiave.
                \item L'utente può ricercare le aste sulla base della categoria.
                \item L’utente dopo aver ricercato per parole chiave può filtrare i risultati sulla base della categoria.
                \item L'utente può visualizzare i dettagli di un'asta.
                \item L'utente può visualizzare il profilo del proprietario dell'asta.
            \end{itemize} %%%
            \item[4]
            \begin{itemize} % Requisiti %
                \item Il venditore può selezionare il tipo di asta da creare.
                \item Il venditore può creare un'asta di tipo "a tempo fisso".
                \item Il venditore può inserire una data di scadenza per l'asta a tempo fisso.
                \item Il venditore può stabilire una soglia minima segreta per l'asta (mai visibile ad altri utenti al di fuori del creatore dell'asta).
                \item Il compratore può visualizzare i dettagli dell'asta.
                \item Il compratore può visualizzare l'attuale offerta più alta dell'asta.
                \item Il compratore può offrire una somma in euro maggiore rispetto alla soglia più alta attualmente raggiunta, entro la data di scadenza.
                \item Il sistema assegna la vincita al compratore che ha offerto la somma più alta entro la data di scadenza dell'asta.
                \item Il sistema dichiara l'asta fallita se nessun compratore ha offerto una somma più alta della soglia minima entro la data di scadenza dell'asta.
                \item Il sistema invia una notifica a tutti i partecipanti dell'asta (ovvero il venditore e tutti gli acquirenti che hanno offerto almeno una volta una somma di denaro) alla chiusura dell'asta.
            \end{itemize} %%%
            \item[7] 
            \begin{itemize} % Requisiti %
                \item Il venditore può creare un'asta di tipo "silenziosa".
                \item Il venditore può inserire una data di scadenza per l'asta silenziosa.
                \item Il compratore può inviare un'offerta segreta al venditore, entro la data di scadenza.
                \item Il compratore non può vedere le offerte degli altri compratori a un'asta.
                \item Il venditore può accettare o rifiutare le singole offerte. Possono essere rifiutate un numero arbitrario di offerte ma può essere accettata solo una sola. \textbf{??????}
                \item Il sistema assegna la vincita dell'asta al compratore la cui offerta è stata accettata dal venditore.
                \item Il sistema dichiara l'asta fallita se il venditore non ha accettato alcuna offerta o non ne è stata avanzata alcuna da parte di un compratore entro la data di scadenza dell'asta.
            \end{itemize} %%%
            \item[8]
            \begin{itemize}            
                \item Il compratore può creare un'asta inversa.
                \item Il compratore può inserire una data di scadenza per l'asta inversa creata.
                \item Il compratore può specificare il prodotto/servizio che richiede nell'asta inversa creata.
                \item Il compratore può opzionalmente inserire/modificare un'immagine del prodotto/servizio specificato nell'asta inversa creata.
                \item Il compratore può inserire un prezzo minimo che è disposto a pagare per il prodotto/servizio dell'asta creata.
                \item Il compratore può inserire/modificare la categoria dell'asta inversa da lui gestita.
                \item Il venditore può offrire una somma in euro minore rispetto alla somma più bassa attualmente raggiunta, entro la data di scadenza.
                \item Il sistema assegna la vincita dell'asta al venditore che ha offerto la somma più bassa al sopraggiungere della data di scadenza dell'asta a cui ha partecipato.
                \item Il sistema dichiara l'asta fallita se nessun venditore ha avanzato un'offerta entro la data di scadenza dell'asta.
            \end{itemize}
        \end{itemize}

     \section{Prototipi}