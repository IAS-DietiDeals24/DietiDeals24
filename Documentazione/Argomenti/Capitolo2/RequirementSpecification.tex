%PDF DI RIFERIMENTO: 04_Modelli di Dominio.pdf, Statecharts.pdf

\chapter*{Requirement Specification}
    \section{Scelte progettuali}

    \section{Individuazione delle responsabilità}

    \section{Class Diagram del dominio del problema}
        Dall'analisi dei requisiti costruiamo il Class Diagram del dominio del problema.\\
        In particolare, il modello concettuale basato sull'analisi dei requisiti permetterà di creare un modello efficace per la gestione delle aste.\\

        %Class Diagram costruito con StarUML
        \begin{figure}[htbp!]
            \centering
                \vspace{2\baselineskip}
                \includegraphics[width=\linewidth]{Immagini/WorkInProgress.pdf}
            \caption{Class Diagram del dominio del problema}
            \label{fig:Class Diagram del dominio del problema}
        \end{figure}

    \section{Dizionario delle classi}
        \begin{tabular}{|C{3.5cm}|L{11.2cm}|}
            \hline
            \multicolumn{1}{|c|}{\textbf{Classe}} & \multicolumn{1}{c|}{\textbf{Descrizione}}\\  
            \hline
                NomeClasse &
                Cosa rappresenta la classe\\
            \hline
                NomeClasse &
                Cosa rappresenta la classe\\
            \hline
                NomeClasse &
                Cosa rappresenta la classe\\
            \hline   
        \end{tabular}
        
    \section{Dizionario delle associazioni}
        \begin{tabular}{|C{3.5cm}|L{11.2cm}|}
            \hline
                \multicolumn{1}{|c|}{\textbf{Associazione}} &
                \multicolumn{1}{c|}{\textbf{Descrizione}}\\            
            \hline
                NomeAssociazione &
                Cosa rappresenta l'associazione\\
            \hline
                NomeAssociazione &
                Cosa rappresenta l'associazione\\
            \hline
                NomeAssociazione &
                Cosa rappresenta l'associazione\\
            \hline
        \end{tabular}

    \section{Statechart}