\chapter{Object Design}
    La progettazione orientata agli oggetti è la fase di sviluppo del software nella quale si definisce il dettaglio implementativo dei vari sottosistemi. \\
    Tali sottosistemi devono soddisfare i requisiti identificati, e sono costruiti sulla base degli oggetti individuati nel dominio del problema, ai quali inevitabilmente saranno aggiunti nuovi oggetti per giungere alla soluzione del problema. \\
    Avendo impiegato un approccio "orientato ai casi d'uso", all'interno del sistema, vi sono tanti \textbf{sottosistemi} quanti sono gli use case. ognuno dei quali si occupa di uno e uno solo use case. \\
    Ogni sottosistema è organizzato in quattro macrosezioni:
    \begin{itemize}
        \item (Client) Lo strato di presentazione dell'interfaccia utente;
        \item (Client) Lo strato di controllo dell'interfaccia utente che costruisce l'interfaccia, controlla la validità dei dati inseriti, gestisce le richieste da inviare al server attraverso le REST API, gestisce le risposte ricevute dal server attraverso le REST API;
        \item (Server) Lo strato di controllo dei dati che controlla la validità dei dati in arrivo, invia i dati da archiviare allo strato di archiviazione, gestisce le risposte da inviare al client attraverso le REST API, gestisce le richieste ricevute dal client attraverso le REST API;
        \item (Server) Lo strato di archiviazione dei dati che si occupa di comunicare con la base di dati per il recupero e il salvataggio dei dati.
    \end{itemize}

    \section{Class Diagram}
        \subsection{Client}
            \begin{figure}[htbp!]
                \includegraphics[width=0.9\linewidth]{Immagini/Diagrammi/Class Diagram/ClassDiagramClient.pdf}
            \caption{Class Diagram client}
            \label{fig:Class Diagram client}
            \end{figure}
        \subsection{Server}
            \begin{figure}[htbp!]
                \includegraphics[width=0.35\linewidth]{Immagini/Diagrammi/Class Diagram/ClassDiagramServer.pdf}
            \caption{Class Diagram server}
            \label{fig:Class Diagram server}
            \end{figure}

    \clearpage
    
    \section{Sequence Diagram}
        \subsection{Client}
        \begin{figure}[htbp!]
            \centering
                \includegraphics[width=0.49\linewidth]{Immagini/Diagrammi/Sequence Diagram/Client Sequence Design/ClientSequenceCreaAstaDesign.pdf}
            \caption{Sequence Diagram di design creazione asta inversa (client)}
            \label{fig:Sequence Diagram di design creazione asta inversa (client)}
        \end{figure}
        \begin{figure}[htbp!]
            \centering
                \includegraphics[width=1\linewidth]{Immagini/Diagrammi/Sequence Diagram/Client Sequence Design/ClientSequenceAccettaOffertaDesign.pdf}
            \caption{Sequence Diagram di design accetta offerta per un'asta silenziosa (client)}
            \label{fig:Sequence Diagram di design accetta offerta per un'asta silenziosa (client)}
        \end{figure}
        \begin{figure}[htbp!]
            \centering
                \includegraphics[width=0.7\linewidth]{Immagini/Diagrammi/Sequence Diagram/Client Sequence Design/ClientSequenceDettagliAstaDesign.pdf}
            \caption{Sequence Diagram di design visualizza dettagli asta (client)}
            \label{fig:Sequence Diagram di design visualizza dettagli asta (client)}
        \end{figure}
        \begin{figure}[htbp!]
            \centering
                \includegraphics[width=0.65\linewidth]{Immagini/Diagrammi/Sequence Diagram/Client Sequence Design/ClientSequenceModificaProfiloDesign.pdf}
            \caption{Sequence Diagram di design modifica dati profilo (client)}
            \label{fig:Sequence Diagram di design modifica dati profilo (client)}
        \end{figure}

    \clearpage
        \subsection{Server}
        \begin{figure}[htbp!]
            \centering
                \includegraphics[width=1\linewidth]{Immagini/Diagrammi/Sequence Diagram/Server Sequence Design/ServerSequenceCreaAstaDesign.pdf}
            \caption{Sequence Diagram di design creazione asta inversa (server)}
            \label{fig:Sequence Diagram di design creazione asta inversa (server)}
        \end{figure}
        \begin{figure}[htbp!]
            \centering
                \includegraphics[width=1\linewidth]{Immagini/Diagrammi/Sequence Diagram/Server Sequence Design/ServerSequenceAccettaOffertaDesign.pdf}
            \caption{Sequence Diagram di design accetta offerta per un'asta silenziosa (server)}
            \label{fig:Sequence Diagram di design accetta offerta per un'asta silenziosa (server)}
        \end{figure}
        \begin{figure}[htbp!]
            \centering
                \includegraphics[width=1\linewidth]{Immagini/Diagrammi/Sequence Diagram/Server Sequence Design/ServerSequenceDettagliAstaDesign.pdf}
            \caption{Sequence Diagram di design visualizza dettagli asta (server)}
            \label{fig:Sequence Diagram di design visualizza dettagli asta (server)}
        \end{figure}
        \begin{figure}[htbp!]
            \centering
                \includegraphics[width=1\linewidth]{Immagini/Diagrammi/Sequence Diagram/Server Sequence Design/ServerSequenceModificaProfiloDesign.pdf}
            \caption{Sequence Diagram di design modifica dati profilo (server)}
            \label{fig:Sequence Diagram di design modifica dati profilo (server)}
        \end{figure}
    \clearpage
    
    \section{Database}
        \subsection{Progettazione concettuale}
            \begin{figure}[htbp!]
                \centering
                    \includegraphics[width=0.7\linewidth]{Immagini/Diagrammi/Class Diagram/ClassDiagramDatabaseRistrutturato.pdf}
                \caption{Schema concettuale del database}
                \label{fig:Schema concettuale del database}
            \end{figure}
            
        \subsection{Progettazione logica}
            \begin{figure}[htbp!]
                \centering
                    \includegraphics[width=0.7\linewidth]{Immagini/Diagrammi/Class Diagram/ClassDiagramDatabaseLogico.pdf}
                \caption{Schema logico del database}
                \label{fig:Schema logico del database}
            \end{figure}
        \subsection{Progettazione fisica}
\begin{lstlisting}[language=SQL, caption=Preparazione ambiente]
    DROP SCHEMA IF EXISTS dd24 CASCADE;
    CREATE SCHEMA dd24;
    SET search_path TO dd24;
\end{lstlisting}
            
\begin{lstlisting}[language=SQL, caption=Relazione categoria asta]
    CREATE TABLE categoria_asta
    (
        nome TEXT NOT NULL,
        CONSTRAINT pk_categoria_asta PRIMARY KEY (nome)
    );
\end{lstlisting}

\begin{lstlisting}[language=SQL, caption=Relazione profilo]
    CREATE TABLE profilo
    (
        nome_utente     TEXT NOT NULL,
        CONSTRAINT pk_profilo PRIMARY KEY (nome_utente),
        area_geografica TEXT,
        biografia       TEXT,
        cognome         TEXT NOT NULL,
        data_nascita    DATE NOT NULL,
        CONSTRAINT chk_data_nascita CHECK (data_nascita <= NOW()),
        genere          TEXT NOT NULL,
        nome            TEXT NOT NULL,
        link_personale  TEXT,
        link_facebook   TEXT,
        link_x          TEXT,
        link_instagram  TEXT,
        link_git_hub    TEXT,
        profile_picture BYTEA
    );
\end{lstlisting}

\begin{lstlisting}[language=SQL, caption=Relazione compratore]
    CREATE TABLE compratore
    (
        id_account          BIGSERIAL NOT NULL,
        CONSTRAINT pk_compratore PRIMARY KEY (id_account),
        email               TEXT      NOT NULL,
        CONSTRAINT uk_compratore_email UNIQUE (email),
        password            TEXT      NOT NULL,
        id_facebook         TEXT,
        id_git_hub          TEXT,
        id_google           TEXT,
        id_x                TEXT,
        profilo_nome_utente TEXT      NOT NULL,
        CONSTRAINT fk_profilo_nome_utente FOREIGN KEY (profilo_nome_utente) REFERENCES profilo (nome_utente) ON UPDATE CASCADE ON DELETE CASCADE
    );
    
    CREATE FUNCTION cleanup_compratore()
        RETURNS TRIGGER AS
    $$
    BEGIN
        DELETE
        FROM offerta_silenziosa
        WHERE compratore_id_account = OLD.id_account;
        DELETE
        FROM offerta_tempo_fisso
        WHERE compratore_id_account = OLD.id_account;
        DELETE
        FROM asta_inversa
        WHERE compratore_id_account = OLD.id_account;
        DELETE
        FROM destinatari
        WHERE account_id_account = OLD.id_account;
        DELETE
        FROM notifica
        WHERE account_id_account = OLD.id_account;
    END
    $$
        LANGUAGE PLPGSQL;
    
    CREATE TRIGGER trg_compratore
        BEFORE DELETE
        ON compratore
        FOR EACH ROW
    EXECUTE FUNCTION cleanup_compratore();
\end{lstlisting}

\begin{lstlisting}[language=SQL, caption=Relazione venditore]
    CREATE TABLE venditore
    (
        id_account          BIGSERIAL NOT NULL,
        CONSTRAINT pk_venditore PRIMARY KEY (id_account),
        email               TEXT      NOT NULL,
        CONSTRAINT uk_venditore_email UNIQUE (email),
        password            TEXT      NOT NULL,
        id_facebook         TEXT,
        id_git_hub          TEXT,
        id_google           TEXT,
        id_x                TEXT,
        profilo_nome_utente TEXT      NOT NULL,
        CONSTRAINT fk_profilo_nome_utente FOREIGN KEY (profilo_nome_utente) REFERENCES profilo (nome_utente) ON UPDATE CASCADE ON DELETE CASCADE
    );
    
    CREATE FUNCTION cleanup_venditore()
        RETURNS TRIGGER AS
    $$
    BEGIN
        DELETE
        FROM offerta_inversa
        WHERE venditore_id_account = OLD.id_account;
        DELETE
        FROM asta_tempo_fisso
        WHERE venditore_id_account = OLD.id_account;
        DELETE
        FROM asta_silenziosa
        WHERE venditore_id_account = OLD.id_account;
        DELETE
        FROM destinatari
        WHERE account_id_account = OLD.id_account;
        DELETE
        FROM notifica
        WHERE account_id_account = OLD.id_account;
    END
    $$
        LANGUAGE PLPGSQL;
    
    CREATE TRIGGER trg_venditore
        BEFORE DELETE
        ON venditore
        FOR EACH ROW
    EXECUTE FUNCTION cleanup_venditore();
\end{lstlisting}

\begin{lstlisting}[language=SQL, caption=Relazione asta inversa]
    CREATE TABLE asta_inversa
    (
        id_asta               BIGSERIAL      NOT NULL,
        CONSTRAINT pk_asta_inversa PRIMARY KEY (id_asta),
        data_scadenza         DATE           NOT NULL,
        CONSTRAINT chk_data_scadenza CHECK (data_scadenza > NOW()),
        descrizione           TEXT           NOT NULL,
        immagine              BYTEA,
        nome                  TEXT           NOT NULL,
        ora_scadenza          TIME           NOT NULL,
        categoria_asta_nome   TEXT           NOT NULL,
        CONSTRAINT fk_categoria_asta_nome FOREIGN KEY (categoria_asta_nome) REFERENCES categoria_asta (nome) ON UPDATE CASCADE ON DELETE CASCADE,
        compratore_id_account BIGINT         NOT NULL,
        CONSTRAINT fk_compratore_id_account FOREIGN KEY (compratore_id_account) REFERENCES compratore (id_account) ON UPDATE CASCADE ON DELETE CASCADE,
        soglia_iniziale       DECIMAL(2, 10) NOT NULL,
        CONSTRAINT chk_soglia_iniziale CHECK (soglia_iniziale >= 0),
        stato                 TEXT           NOT NULL
    );
    
    CREATE FUNCTION cleanup_asta_inversa()
        RETURNS TRIGGER AS
    $$
    BEGIN
        DELETE
        FROM notifica
        WHERE asta_id_asta = OLD.id_asta;
    END
    $$
        LANGUAGE PLPGSQL;
    
    CREATE TRIGGER trg_asta_inversa
        BEFORE DELETE
        ON asta_inversa
        FOR EACH ROW
    EXECUTE FUNCTION cleanup_asta_inversa();
\end{lstlisting}

\begin{lstlisting}[language=SQL, caption=Relazione asta silenziosa]
    CREATE TABLE asta_silenziosa
    (
        id_asta              BIGSERIAL NOT NULL,
        CONSTRAINT pk_asta_silenziosa PRIMARY KEY (id_asta),
        data_scadenza        DATE      NOT NULL,
        CONSTRAINT chk_data_scadenza CHECK (data_scadenza > NOW()),
        descrizione          TEXT      NOT NULL,
        immagine             BYTEA,
        nome                 TEXT      NOT NULL,
        ora_scadenza         TIME      NOT NULL,
        categoria_asta_nome  TEXT      NOT NULL,
        CONSTRAINT fk_categoria_asta_nome FOREIGN KEY (categoria_asta_nome) REFERENCES categoria_asta (nome) ON UPDATE CASCADE ON DELETE CASCADE,
        venditore_id_account BIGINT    NOT NULL,
        CONSTRAINT fk_venditore_id_account FOREIGN KEY (venditore_email) REFERENCES venditore (id_account) ON UPDATE CASCADE ON DELETE CASCADE,
        stato                TEXT      NOT NULL
    );
    
    CREATE FUNCTION cleanup_asta_silenziosa()
        RETURNS TRIGGER AS
    $$
    BEGIN
        DELETE
        FROM notifica
        WHERE asta_id_asta = OLD.id_asta;
    END
    $$
        LANGUAGE PLPGSQL;
    
    CREATE TRIGGER trg_asta_silenziosa
        BEFORE DELETE
        ON asta_silenziosa
        FOR EACH ROW
    EXECUTE FUNCTION cleanup_asta_silenziosa();
\end{lstlisting}

\begin{lstlisting}[language=SQL, caption=Relazione asta a tempo fisso]
    CREATE TABLE asta_tempo_fisso
    (
        id_asta              BIGSERIAL NOT NULL,
        CONSTRAINT pk_asta_tempo_fisso PRIMARY KEY (id_asta),
        data_scadenza        DATE           NOT NULL,
        CONSTRAINT chk_data_scadenza CHECK (data_scadenza > NOW()),
        descrizione          TEXT           NOT NULL,
        immagine             BYTEA,
        nome                 TEXT           NOT NULL,
        ora_scadenza         TIME           NOT NULL,
        categoria_asta_nome  TEXT           NOT NULL,
        CONSTRAINT fk_categoria_asta_nome FOREIGN KEY (categoria_asta_nome) REFERENCES categoria_asta (nome) ON UPDATE CASCADE ON DELETE CASCADE,
        venditore_id_account BIGINT         NOT NULL,
        CONSTRAINT fk_venditore_id_account FOREIGN KEY (venditore_id_account) REFERENCES venditore (id_account) ON UPDATE CASCADE ON DELETE CASCADE,
        soglia_minima        DECIMAL(2, 10) NOT NULL,
        CONSTRAINT chk_soglia_minima CHECK (soglia_minima >= 0),
        stato                TEXT           NOT NULL
    );
    
    CREATE FUNCTION cleanup_asta_tempo_fisso()
        RETURNS TRIGGER AS
    $$
    BEGIN
        DELETE
        FROM notifica
        WHERE asta_id_asta = OLD.id_asta;
    END
    $$
        LANGUAGE PLPGSQL;
    
    CREATE TRIGGER trg_asta_tempo_fisso
        BEFORE DELETE
        ON asta_tempo_fisso
        FOR EACH ROW
    EXECUTE FUNCTION cleanup_asta_tempo_fisso();
\end{lstlisting}

\begin{lstlisting}[language=SQL, caption=Relazione notifica]
    CREATE TABLE notifica
    (
        id_notifica        BIGSERIAL NOT NULL,
        CONSTRAINT pk_notifica PRIMARY KEY (id_notifica),
        data_invio         DATE      NOT NULL,
        CONSTRAINT chk_data_invio CHECK (data_invio <= NOW()),
        messaggio          TEXT      NOT NULL,
        ora_invio          TIME      NOT NULL,
        asta_id_asta       BIGINT    NOT NULL,
        account_id_account BIGINT    NOT NULL
    );
    
    CREATE FUNCTION chk_asta_id_asta()
        RETURNS TRIGGER AS
    $$
    BEGIN
        IF
            NOT EXISTS (SELECT id_asta FROM asta_inversa WHERE id_asta = NEW.asta_id_asta) AND
            NOT EXISTS (SELECT id_asta FROM asta_silenziosa WHERE id_asta = NEW.asta_id_asta) AND
            NOT EXISTS (SELECT id_asta FROM asta_tempo_fisso WHERE id_asta = NEW.asta_id_asta) THEN
            RAISE EXCEPTION 'L''identificativo dell''asta del record inserito non referenzia un''asta esistente';
        ELSE
            RETURN NEW;
        END IF;
    END;
    $$
        LANGUAGE PLPGSQL;
    
    CREATE TRIGGER trg_asta_id_asta
        BEFORE INSERT OR
            UPDATE OF asta_id_asta
        ON notifica
        FOR EACH ROW
    EXECUTE FUNCTION chk_asta_id_asta();
    
    CREATE FUNCTION chk_account_id_account()
        RETURNS TRIGGER AS
    $$
    BEGIN
        IF
            NOT EXISTS (SELECT id_account FROM compratore WHERE id_account = NEW.id_account) AND
            NOT EXISTS (SELECT id_account FROM venditore WHERE id_account = NEW.id_account) THEN
            RAISE EXCEPTION 'L''identificativo dell''account del record inserito non referenzia un account esistente';
        ELSE
            RETURN NEW;
        END IF;
    END;
    $$
        LANGUAGE PLPGSQL;
    
    CREATE TRIGGER trg_account_id_account
        BEFORE INSERT OR
            UPDATE OF account_id_account
        ON notifica
        FOR EACH ROW
    EXECUTE FUNCTION chk_account_id_account();
\end{lstlisting}

\begin{lstlisting}[language=SQL, caption=Relazione destinatari]
    CREATE TABLE destinatari
    (
        notifica_id_notifica BIGINT NOT NULL,
        CONSTRAINT fk_notifica_id_notifica FOREIGN KEY (notifica_id_notifica) REFERENCES notifica (id_notifica) ON UPDATE CASCADE ON DELETE CASCADE,
        account_id_account   BIGINT NOT NULL,
        CONSTRAINT pk_destinatari PRIMARY KEY (notifica_id_notifica, account_id_account)
    );
    
    CREATE TRIGGER trg_account_id_account
        BEFORE INSERT OR
            UPDATE OF account_id_account
        ON destinatari
        FOR EACH ROW
    EXECUTE FUNCTION chk_account_id_account();
\end{lstlisting}

\begin{lstlisting}[language=SQL, caption=Relazione offerta inversa]
    CREATE TABLE offerta_inversa
    (
        id_offerta           BIGSERIAL      NOT NULL,
        CONSTRAINT pk_offerta_inversa PRIMARY KEY (id_offerta),
        data_invio           DATE           NOT NULL,
        CONSTRAINT chk_data_invio CHECK (data_invio <= NOW()),
        ora_invio            TIME           NOT NULL,
        valore               DECIMAL(2, 10) NOT NULL,
        CONSTRAINT chk_valore CHECK (valore > 0),
        venditore_id_account BIGINT         NOT NULL,
        CONSTRAINT fk_venditore_id_account FOREIGN KEY (venditore_id_account) REFERENCES venditore (id_account) ON UPDATE CASCADE ON DELETE CASCADE,
        asta_inversa_id_asta BIGINT         NOT NULL,
        CONSTRAINT fk_asta_inversa_id_asta FOREIGN KEY (asta_inversa_id_asta) REFERENCES asta_inversa (id_asta) ON UPDATE CASCADE ON DELETE CASCADE
    );
\end{lstlisting}

\begin{lstlisting}[language=SQL, caption=Relazione offerta silenziosa]
    CREATE TABLE offerta_silenziosa
    (
        id_offerta              BIGSERIAL      NOT NULL,
        CONSTRAINT pk_offerta_silenziosa PRIMARY KEY (id_offerta),
        data_invio              DATE           NOT NULL,
        CONSTRAINT chk_data_invio CHECK (data_invio <= NOW()),
        ora_invio               TIME           NOT NULL,
        valore                  DECIMAL(2, 10) NOT NULL,
        CONSTRAINT chk_valore CHECK (valore > 0),
        compratore_id_account   BIGINT         NOT NULL,
        CONSTRAINT fk_compratore_id_account FOREIGN KEY (compratore_id_account) REFERENCES compratore (id_account) ON UPDATE CASCADE ON DELETE CASCADE,
        stato                   TEXT           NOT NULL,
        asta_silenziosa_id_asta BIGINT         NOT NULL,
        CONSTRAINT fk_asta_silenziosa_id_asta FOREIGN KEY (asta_silenziosa_id_asta) REFERENCES asta_silenziosa (id_asta) ON UPDATE CASCADE ON DELETE CASCADE
    );
\end{lstlisting}

\begin{lstlisting}[language=SQL, caption=Relazione offerta a tempo fisso]
    CREATE TABLE offerta_tempo_fisso
    (
        id_offerta               BIGSERIAL      NOT NULl,
        CONSTRAINT pk_offerta_tempo_fisso PRIMARY KEY (id_offerta),
        data_invio               DATE           NOT NULL,
        CONSTRAINT chk_data_invio CHECK (data_invio <= NOW()),
        ora_invio                TIME           NOT NULL,
        valore                   DECIMAL(2, 10) NOT NULL,
        CONSTRAINT chk_valore CHECK (valore > 0),
        compratore_id_account    BIGINT         NOT NULL,
        CONSTRAINT fk_compratore_id_account FOREIGN KEY (compratore_id_account) REFERENCES compratore (id_account) ON UPDATE CASCADE ON DELETE CASCADE,
        asta_tempo_fisso_id_asta BIGINT         NOT NULL,
        CONSTRAINT fk_asta_tempo_fisso_id_asta FOREIGN KEY (asta_tempo_fisso_id_asta) REFERENCES asta_tempo_fisso (id_asta) ON UPDATE CASCADE ON DELETE CASCADE
    );
\end{lstlisting}