%PDF DI RIFERIMENTO: 03_Requirement Engineering.pdf

\chapter{Requirement Elicitation}
    \section{Obiettivo}
        A seguito di una intervista agli stakeholders per carpire informazioni sul dominio del problema, nonché le funzionalità e le qualità che il software debba possedere per risultare conforme ai bisogni dell'utente, si elencano di seguito tutti i requisiti individuati:
    \section{Introduzione al sistema} %Requisiti sistema. 1 pagina
        \subsection{Requisiti funzionali: utente non registrato}
        \begin{tabular}{|p{0.9\textwidth}|}
            \hline
            \multicolumn{1}{|l|}{\cellcolor{head}\textbf{REQ-NRU-01}} \\
            \hline
            L'utente che non ha effettuato l'accesso può registrarsi con email correttamente formattata e password. \\
            \hline
        \end{tabular} \smallskip \\
        \begin{tabular}{|p{0.9\textwidth}|}
            \hline
            \multicolumn{1}{|l|}{\cellcolor{head}\textbf{REQ-NRU-02}} \\
            \hline
            L'utente che non ha effettuato l'accesso può registrarsi con un account Facebook già esistente. \\
            \hline
        \end{tabular} \smallskip \\
        \begin{tabular}{|p{0.9\textwidth}|}
            \hline
            \multicolumn{1}{|l|}{\cellcolor{head}\textbf{REQ-NRU-03}} \\
            \hline
            L'utente che non ha effettuato l'accesso può registrarsi con un account Google già esistente. \\
            \hline
        \end{tabular} \smallskip \\
        \begin{tabular}{|p{0.9\textwidth}|}
            \hline
            \multicolumn{1}{|l|}{\cellcolor{head}\textbf{REQ-NRU-04}} \\
            \hline
            L'utente che non ha effettuato l'accesso può selezionare se il suo nuovo account è di tipo compratore o venditore. \\
            \hline
        \end{tabular} \smallskip \\
        \begin{tabular}{|p{0.9\textwidth}|}
            \hline
            \multicolumn{1}{|l|}{\cellcolor{head}\textbf{REQ-NRU-05}} \\
            \hline
            L'utente che non ha effettuato l'accesso può creare uno e un solo account di tipo compratore con la stessa email. \\
            \hline
        \end{tabular} \smallskip \\
        \begin{tabular}{|p{0.9\textwidth}|}
            \hline
            \multicolumn{1}{|l|}{\cellcolor{head}\textbf{REQ-NRU-06}} \\
            \hline
            L'utente che non ha effettuato l'accesso può creare uno e un solo account di tipo venditore con la stessa email. \\
            \hline
        \end{tabular} \smallskip \\
        \begin{tabular}{|p{0.9\textwidth}|}
            \hline
            \multicolumn{1}{|l|}{\cellcolor{head}\textbf{REQ-NRU-07}} \\
            \hline
            L'utente che non ha effettuato l'accesso può creare un profilo condiviso tra l'account venditore e l'account compratore con la stessa email. La creazione del profilo avviene subito dopo la fase di registrazione account. \\
            \hline
        \end{tabular} \smallskip \\
        \begin{tabular}{|p{0.9\textwidth}|}
            \hline
            \multicolumn{1}{|l|}{\cellcolor{head}\textbf{REQ-NRU-08}} \\
            \hline
            L'utente che non ha effettuato l'accesso può accedere a un account con l'email e password utilizzati nella registrazione. \\
            \hline
        \end{tabular} \smallskip \\
        \begin{tabular}{|p{0.9\textwidth}|}
            \hline
            \multicolumn{1}{|l|}{\cellcolor{head}\textbf{REQ-NRU-09}} \\
            \hline
            L'utente che non ha effettuato l'accesso può accedere con un account Facebook già esistente. \\
            \hline
        \end{tabular} \smallskip \\
        \begin{tabular}{|p{0.9\textwidth}|}
            \hline
            \multicolumn{1}{|l|}{\cellcolor{head}\textbf{REQ-NRU-10}} \\
            \hline
            L'utente che non ha effettuato l'accesso può accedere con un account Google già esistente. \\
            \hline
        \end{tabular} \smallskip \\
        \subsection{Requisiti funzionali: utente registrato}
        \begin{tabular}{|p{0.9\textwidth}|}
            \hline
            \multicolumn{1}{|l|}{\cellcolor{head}\textbf{REQ-RU-01}} \\
            \hline
            L'utente che ha effettuato l'accesso può visualizzare il suo profilo. \\
            \hline
        \end{tabular} \smallskip \\
        \begin{tabular}{|p{0.9\textwidth}|}
            \hline
            \multicolumn{1}{|l|}{\cellcolor{head}\textbf{REQ-RU-02}} \\
            \hline
            L'utente che ha effettuato l'accesso può modificare i dati sul suo profilo, come biografia, link al proprio sito, link social e area geografica. \\
            \hline
        \end{tabular} \smallskip \\
        \begin{tabular}{|p{0.9\textwidth}|}
            \hline
            \multicolumn{1}{|l|}{\cellcolor{head}\textbf{REQ-RU-03}} \\
            \hline
            L'utente che ha effettuato l'accesso può visualizzare le aste che ha creato. \\
            \hline
        \end{tabular} \smallskip \\
        \begin{tabular}{|p{0.9\textwidth}|}
            \hline
            \multicolumn{1}{|l|}{\cellcolor{head}\textbf{REQ-RU-04}} \\
            \hline
            L'utente che ha effettuato l'accesso può eliminare le aste che ha creato. \\
            \hline
        \end{tabular} \smallskip \\
        \begin{tabular}{|p{0.9\textwidth}|}
            \hline
            \multicolumn{1}{|l|}{\cellcolor{head}\textbf{REQ-RU-05}} \\
            \hline
            L'utente che ha effettuato l'accesso può visualizzare le aste a cui ha partecipato. \\
            \hline
        \end{tabular} \smallskip \\
        \begin{tabular}{|p{0.9\textwidth}|}
            \hline
            \multicolumn{1}{|l|}{\cellcolor{head}\textbf{REQ-RU-06}} \\
            \hline
            L'utente che ha effettuato l'accesso può visualizzare l'elenco delle offerte proposte alle aste da lui create. \\
            \hline
        \end{tabular} \smallskip \\
        \begin{tabular}{|p{0.9\textwidth}|}
            \hline
            \multicolumn{1}{|l|}{\cellcolor{head}\textbf{REQ-RU-07}} \\
            \hline
            L'utente che ha effettuato l'accesso può visualizzare le proprie notifiche. \\
            \hline
        \end{tabular} \smallskip \\
        \subsection{Requisiti funzionali: utente generico}
        \begin{tabular}{|p{0.9\textwidth}|}
            \hline
            \multicolumn{1}{|l|}{\cellcolor{head}\textbf{REQ-USR-01}} \\
            \hline
            L'utente (sia che abbia effettuato l'accesso, sia che non l'abbia effettuato) può ricercare le aste attraverso parole chiave. \\
            \hline
        \end{tabular} \smallskip \\
        \begin{tabular}{|p{0.9\textwidth}|}
            \hline
            \multicolumn{1}{|l|}{\cellcolor{head}\textbf{REQ-USR-02}} \\
            \hline
            L'utente (sia che abbia effettuato l'accesso, sia che non l'abbia effettuato) può ricercare le aste sulla base della categoria. \\
            \hline
        \end{tabular} \smallskip \\
        \begin{tabular}{|p{0.9\textwidth}|}
            \hline
            \multicolumn{1}{|l|}{\cellcolor{head}\textbf{REQ-USR-03}} \\
            \hline
            L'utente (sia che abbia effettuato l'accesso, sia che non l'abbia effettuato) può ricercare le aste attraverso parole chiave e filtrando sulla base della categoria. \\
            \hline
        \end{tabular} \smallskip \\
        \begin{tabular}{|p{0.9\textwidth}|}
            \hline
            \multicolumn{1}{|l|}{\cellcolor{head}\textbf{REQ-USR-04}} \\
            \hline
            L'utente (sia che abbia effettuato l'accesso, sia che non l'abbia effettuato) può visualizzare i dettagli di un'asta. \\
            \hline
        \end{tabular} \smallskip \\
        \begin{tabular}{|p{0.9\textwidth}|}
            \hline
            \multicolumn{1}{|l|}{\cellcolor{head}\textbf{REQ-USR-05}} \\
            \hline
            L'utente (sia che abbia effettuato l'accesso, sia che non l'abbia effettuato) può visualizzare il profilo del proprietario dell'asta. \\
            \hline
        \end{tabular} \smallskip \\
        \begin{tabular}{|p{0.9\textwidth}|}
            \hline
            \multicolumn{1}{|l|}{\cellcolor{head}\textbf{REQ-USR-06}} \\
            \hline
            L'utente (sia che abbia effettuato l'accesso, sia che non l'abbia effettuato) può visualizzare le aste attive. \\
            \hline
        \end{tabular} \smallskip \\
        \begin{tabular}{|p{0.9\textwidth}|}
            \hline
            \multicolumn{1}{|l|}{\cellcolor{head}\textbf{REQ-USR-07}} \\
            \hline
            L'utente (sia che abbia effettuato l'accesso, sia che non l'abbia effettuato) può visualizzare l'attuale offerta più alta dell'asta di tipo "a tempo fisso". \\
            \hline
        \end{tabular} \smallskip \\
        \begin{tabular}{|p{0.9\textwidth}|}
            \hline
            \multicolumn{1}{|l|}{\cellcolor{head}\textbf{REQ-USR-08}} \\
            \hline
            L'utente (sia che abbia effettuato l'accesso, sia che non l'abbia effettuato) può visualizzare l'attuale offerta più bassa dell'asta di tipo "inversa". \\
            \hline
        \end{tabular} \smallskip \\
        \begin{tabular}{|p{0.9\textwidth}|}
            \hline
            \multicolumn{1}{|l|}{\cellcolor{head}\textbf{REQ-USR-09}} \\
            \hline
            L'utente (sia che abbia effettuato l'accesso, sia che non l'abbia effettuato) può accedere ad una sezione di aiuto per leggere istruzioni sull'utilizzo dell'applicativo. \\
            \hline
        \end{tabular} \smallskip \\
        \subsection{Requisiti funzionali: compratore}
        \begin{tabular}{|p{0.9\textwidth}|}
            \hline
            \multicolumn{1}{|l|}{\cellcolor{head}\textbf{REQ-BUY-01}} \\
            \hline
            Il compratore può creare aste di acquisto di un prodotto/servizio. L'asta risulterà attiva (cioè sarà possibile presentare offerte) fin da subito. \\
            \hline
        \end{tabular} \smallskip \\
        \begin{tabular}{|p{0.9\textwidth}|}
            \hline
            \multicolumn{1}{|l|}{\cellcolor{head}\textbf{REQ-BUY-02}} \\
            \hline
            Il compratore può creare un'asta di tipo "inversa". \\
            \hline
        \end{tabular} \smallskip \\
        \begin{tabular}{|p{0.9\textwidth}|}
            \hline
            \multicolumn{1}{|l|}{\cellcolor{head}\textbf{REQ-BUY-03}} \\
            \hline
            Il compratore di un'asta di tipo "inversa" può inserire, al momento della creazione dell'asta, una data e ora di scadenza, un titolo, una descrizione, una categoria, un prezzo di partenza che è disposto a pagare e opzionalmente una o più fotografie per l'asta da lui creata. \\
            \hline
        \end{tabular} \smallskip \\
        \begin{tabular}{|p{0.9\textwidth}|}
            \hline
            \multicolumn{1}{|l|}{\cellcolor{head}\textbf{REQ-BUY-04}} \\
            \hline
            Il compratore di un'asta di tipo "inversa" può modificare la data e ora di scadenza, il titolo, la descrizione, la categoria e le fotografie (se aggiunte) per l'asta da lui creata. \\
            \hline
        \end{tabular} \smallskip \\
        \begin{tabular}{|p{0.9\textwidth}|}
            \hline
            \multicolumn{1}{|l|}{\cellcolor{head}\textbf{REQ-BUY-05}} \\
            \hline
            Il compratore che vuole acquistare quel particolare prodotto/servizio può presentare offerte per le aste di tipo "a tempo fisso" o "silenziosa" entro la data e ora di scadenza dell'asta. Tale offerta è una somma in euro sempre maggiore di 0, e per l'asta di tipo "a tempo fisso" deve anche essere maggiore dell'attuale offerta più alta.  \\
            \hline
        \end{tabular} \smallskip \\
        \subsection{Requisiti funzionali: venditore}
        \begin{tabular}{|p{0.9\textwidth}|}
            \hline
            \multicolumn{1}{|l|}{\cellcolor{head}\textbf{REQ-SEL-01}} \\
            \hline
            Il venditore può creare aste di vendita di un prodotto/servizio. L'asta risulterà attiva (cioè sarà possibile presentare offerte) fin da subito. \\
            \hline
        \end{tabular} \smallskip \\
        \begin{tabular}{|p{0.9\textwidth}|}
            \hline
            \multicolumn{1}{|l|}{\cellcolor{head}\textbf{REQ-SEL-02}} \\
            \hline
            Il venditore può selezionare il tipo di asta da creare. L'asta può essere di tipo "a tempo fisso" o "silenziosa". \\
            \hline
        \end{tabular} \smallskip \\
        \begin{tabular}{|p{0.9\textwidth}|}
            \hline
            \multicolumn{1}{|l|}{\cellcolor{head}\textbf{REQ-SEL-03}} \\
            \hline
            Il venditore di un'asta di qualsiasi tipo può inserire, al momento della creazione dell'asta, una data e ora di scadenza, un titolo, una descrizione, una categoria e opzionalmente una o più fotografie per l'asta da lui creata. Per l'asta di tipo "a tempo fisso", può anche specificare un prezzo minimo da raggiungere. \\
            \hline
        \end{tabular} \smallskip \\
        \begin{tabular}{|p{0.9\textwidth}|}
            \hline
            \multicolumn{1}{|l|}{\cellcolor{head}\textbf{REQ-SEL-04}} \\
            \hline
            Il venditore di un'asta di qualsiasi tipo può modificare la data e ora di scadenza, il titolo, la descrizione, la categoria e le fotografie (se aggiunte) per l'asta da lui creata. \\
            \hline
        \end{tabular} \smallskip \\
        \begin{tabular}{|p{0.9\textwidth}|}
            \hline
            \multicolumn{1}{|l|}{\cellcolor{head}\textbf{REQ-SEL-05}} \\
            \hline
            Il venditore in grado di fornire quel particolare prodotto/servizio può presentare offerte per le aste di tipo "inversa" entro la data e ora di scadenza dell'asta. Tale offerta è una somma in euro minore rispetto alla somma più bassa attualmente raggiunta (tale valore deve essere maggiore di 0). \\
            \hline
        \end{tabular} \smallskip \\
        \subsection{Requisiti funzionali: sistema}
        \begin{tabular}{|p{0.9\textwidth}|}
            \hline
            \multicolumn{1}{|l|}{\cellcolor{head}\textbf{REQ-SYS-01}} \\
            \hline
            Il sistema assegna la vincita al compratore di un asta di tipo "a tempo fisso" che ha offerto la somma più alta entro la data e ora di scadenza dell'asta. \\
            \hline
        \end{tabular} \smallskip \\
        \begin{tabular}{|p{0.9\textwidth}|}
            \hline
            \multicolumn{1}{|l|}{\cellcolor{head}\textbf{REQ-SYS-02}} \\
            \hline
            Il sistema dichiara l'asta fallita se nessun compratore di un asta di tipo "a tempo fisso" ha offerto una somma più alta della soglia minima entro la data e ora di scadenza dell'asta. \\
            \hline
        \end{tabular} \smallskip \\
        \begin{tabular}{|p{0.9\textwidth}|}
            \hline
            \multicolumn{1}{|l|}{\cellcolor{head}\textbf{REQ-SYS-03}} \\
            \hline
            Il sistema dichiara l'asta fallita se non è stata avanzata alcuna offerta da parte di un compratore di un'asta di tipo "a tempo fisso" entro la data e ora di scadenza dell'asta. \\
            \hline
        \end{tabular} \smallskip \\
        \begin{tabular}{|p{0.9\textwidth}|}
            \hline
            \multicolumn{1}{|l|}{\cellcolor{head}\textbf{REQ-SYS-04}} \\
            \hline
            Il sistema mostra l'offerta di un'asta di tipo "silenziosa" solo al venditore di tale asta e a colui che ha proposto l'offerta. \\
            \hline
        \end{tabular} \smallskip \\
        \begin{tabular}{|p{0.9\textwidth}|}
            \hline
            \multicolumn{1}{|l|}{\cellcolor{head}\textbf{REQ-SYS-05}} \\
            \hline
            Il sistema assegna la vincita dell'asta al compratore di un'asta di tipo "silenziosa" la cui offerta è stata accettata dal venditore dell'asta di tipo "silenziosa". \\
            \hline
        \end{tabular} \smallskip \\
        \begin{tabular}{|p{0.9\textwidth}|}
            \hline
            \multicolumn{1}{|l|}{\cellcolor{head}\textbf{REQ-SYS-06}} \\
            \hline
            Il sistema dichiara l'asta fallita se il venditore di un'asta di tipo "silenziosa" non ha accettato alcuna offerta entro la data e ora di scadenza dell'asta. \\
            \hline
        \end{tabular} \smallskip \\
        \begin{tabular}{|p{0.9\textwidth}|}
            \hline
            \multicolumn{1}{|l|}{\cellcolor{head}\textbf{REQ-SYS-07}} \\
            \hline
            Il sistema dichiara l'asta fallita se non è stata avanzata alcuna offerta da parte di un compratore di un'asta di tipo "silenziosa" entro la data e ora di scadenza dell'asta. \\
            \hline
        \end{tabular} \smallskip \\
        \begin{tabular}{|p{0.9\textwidth}|}
            \hline
            \multicolumn{1}{|l|}{\cellcolor{head}\textbf{REQ-SYS-08}} \\
            \hline
            Il sistema invia una notifica di proposta offerta al venditore dell'asta di tipo "silenziosa" che ha ricevuto una nuova offerta alla propria asta. \\
            \hline
        \end{tabular} \smallskip \\
        \begin{tabular}{|p{0.9\textwidth}|}
            \hline
            \multicolumn{1}{|l|}{\cellcolor{head}\textbf{REQ-SYS-09}} \\
            \hline
            Il sistema invia una notifica di rifiuto dell'offerta al partecipante dell'asta di tipo "silenziosa" la cui offerta è stata rifiutata dal venditore dell'asta. \\
            \hline
        \end{tabular} \smallskip \\
        \begin{tabular}{|p{0.9\textwidth}|}
            \hline
            \multicolumn{1}{|l|}{\cellcolor{head}\textbf{REQ-SYS-10}} \\
            \hline
            Il sistema assegna la vincita dell'asta di tipo "inversa" al venditore che ha offerto la somma più bassa al sopraggiungere della data e ora di scadenza dell'asta a cui ha partecipato. \\
            \hline
        \end{tabular} \smallskip \\
        \begin{tabular}{|p{0.9\textwidth}|}
            \hline
            \multicolumn{1}{|l|}{\cellcolor{head}\textbf{REQ-SYS-11}} \\
            \hline
            Il sistema dichiara l'asta di tipo "inversa" fallita se non è stata avanzata alcuna offerta da parte di un venditore entro la data e ora di scadenza dell'asta. \\
            \hline
        \end{tabular} \smallskip \\
        \begin{tabular}{|p{0.9\textwidth}|}
            \hline
            \multicolumn{1}{|l|}{\cellcolor{head}\textbf{REQ-SYS-12}} \\
            \hline
            Il sistema controlla che l'email abbia il formato corretto (ovvero sia del tipo x+@y+.z+) e che la password sia sicura (ovvero sia lunga almeno 8 caratteri e contenga, nel complesso, almeno una lettera maiuscola [A-Z], una lettere minuscola [a-z], un numero [0-9] e un carattere speciale [!@\#\$\%\^{}\&*]). \\
            \hline
        \end{tabular} \smallskip \\
        \begin{tabular}{|p{0.9\textwidth}|}
            \hline
            \multicolumn{1}{|l|}{\cellcolor{head}\textbf{REQ-SYS-13}} \\
            \hline
            Il sistema invia una notifica di chiusura dell'asta a tutti i partecipanti dell'asta di tipo "a tempo fisso" (ovvero il venditore e tutti i compratori che hanno offerto almeno una volta una somma di denaro). \\
            \hline
        \end{tabular} \smallskip \\
        \begin{tabular}{|p{0.9\textwidth}|}
            \hline
            \multicolumn{1}{|l|}{\cellcolor{head}\textbf{REQ-SYS-14}} \\
            \hline
            Il sistema invia una notifica di chiusura dell'asta a tutti i partecipanti dell'asta di tipo "silenziosa" (ovvero a tutti i compratori che hanno offerto almeno una volta una somma di denaro) entro 10 secondi dal momento in cui è stata accettata un'offerta. Tale notifica indicherà se la propria offerta è stata accettata o rifiutata. \\
            \hline
        \end{tabular} \smallskip \\
        \begin{tabular}{|p{0.9\textwidth}|}
            \hline
            \multicolumn{1}{|l|}{\cellcolor{head}\textbf{REQ-SYS-15}} \\
            \hline
            Il sistema invia una notifica di chiusura dell'asta a tutti i partecipanti dell'asta di tipo "inversa" (ovvero il compratore e tutti i venditori che hanno offerto almeno una volta una somma di denaro). \\
            \hline
        \end{tabular} \smallskip \\
        \subsection{Requisiti non funzionali}
        \begin{tabular}{|p{0.9\textwidth}|}
            \hline
            \multicolumn{1}{|l|}{\cellcolor{head}\textbf{REQ-A}} \\
            \hline
            Il sistema deve far recapitare le notifiche agli utenti in meno di 10 secondi. \\
            \hline
        \end{tabular} \smallskip \\
        \begin{tabular}{|p{0.9\textwidth}|}
            \hline
            \multicolumn{1}{|l|}{\cellcolor{head}\textbf{REQ-B}} \\
            \hline
            Il sistema deve elaborare quali aste hanno raggiunto la data di scadenza entro 5 secondi dalla loro conclusione. \\
            \hline
        \end{tabular} \smallskip \\
        \begin{tabular}{|p{0.9\textwidth}|}
            \hline
            \multicolumn{1}{|l|}{\cellcolor{head}\textbf{REQ-C}} \\
            \hline
            Le operazioni di rete effettuate dal sistema non devono richiedere più di un secondo. \\
            \hline
        \end{tabular} \smallskip \\
        \begin{tabular}{|p{0.9\textwidth}|}
            \hline
            \multicolumn{1}{|l|}{\cellcolor{head}\textbf{REQ-D}} \\
            \hline
            Il sistema non deve utilizzare più di 500 Mb della memoria della macchina che lo esegue. \\
            \hline
        \end{tabular} \smallskip \\
        